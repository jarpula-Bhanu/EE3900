\documentclass[journal,12pt,twocolumn]{IEEEtran}
\usepackage{setspace}
\usepackage{gensymb}
\usepackage{caption}
%\usepackage{multirow}
%\usepackage{multicolumn}
%\usepackage{subcaption}
%\doublespacing
\singlespacing
\usepackage{csvsimple}
\usepackage{amssymb}
\usepackage{amsmath}
\usepackage{multicol}
%\usepackage{enumerate}
\usepackage{amssymb}
%\usepackage{graphicx}
\usepackage{newfloat}
%\usepackage{syntax}
\usepackage{listings}
\usepackage{color}
\usepackage{tikz}
\usepackage{graphicx}
\usetikzlibrary{shapes,arrows}

%\usepackage{graphicx}
%\usepackage{amssymb}
%\usepackage{relsize}
%\usepackage[cmex10]{amsmath}
%\usepackage{mathtools}
%\usepackage{amsthm}
%\interdisplaylinepenalty=2500
%\savesymbol{iint}
%\usepackage{txfonts}
%\restoresymbol{TXF}{iint}
%\usepackage{wasysym}
\usepackage{amsthm}
\usepackage{mathrsfs}
\usepackage{txfonts}
\usepackage{stfloats}
\usepackage{cite}
\usepackage{cases}
\usepackage{mathtools}
\usepackage{caption}
\usepackage{enumerate}	
\usepackage{enumitem}
\usepackage{amsmath}
%\usepackage{xtab}
\usepackage{longtable}
\usepackage{multirow}
%\usepackage{algorithm}
%\usepackage{algpseudocode}
\usepackage{enumitem}
\usepackage{mathtools}
\usepackage{hyperref}
%\usepackage[framemethod=tikz]{mdframed}
\usepackage{listings}
    %\usepackage[latin1]{inputenc}                                 %%
    \usepackage{color}                                            %%
    \usepackage{array}                                            %%
    \usepackage{longtable}                                        %%
    \usepackage{calc}                                             %%
    \usepackage{multirow}                                         %%
    \usepackage{hhline}                                           %%
    \usepackage{ifthen}                                           %%
  %optionally (for landscape tables embedded in another document): %%
    \usepackage{lscape}     


\usepackage{url}
\def\UrlBreaks{\do\/\do-}


%\usepackage{stmaryrd}


%\usepackage{wasysym}
%\newcounter{MYtempeqncnt}
\DeclareMathOperator*{\Res}{Res}
%\renewcommand{\baselinestretch}{2}
\renewcommand\thesection{\arabic{section}}
\renewcommand\thesubsection{\thesection.\arabic{subsection}}
\renewcommand\thesubsubsection{\thesubsection.\arabic{subsubsection}}

\renewcommand\thesectiondis{\arabic{section}}
\renewcommand\thesubsectiondis{\thesectiondis.\arabic{subsection}}
\renewcommand\thesubsubsectiondis{\thesubsectiondis.\arabic{subsubsection}}

% correct bad hyphenation here
\hyphenation{op-tical net-works semi-conduc-tor}

%\lstset{
%language=C,
%frame=single, 
%breaklines=true
%}

%\lstset{
	%%basicstyle=\small\ttfamily\bfseries,
	%%numberstyle=\small\ttfamily,
	%language=Octave,
	%backgroundcolor=\color{white},
	%%frame=single,
	%%keywordstyle=\bfseries,
	%%breaklines=true,
	%%showstringspaces=false,
	%%xleftmargin=-10mm,
	%%aboveskip=-1mm,
	%%belowskip=0mm
%}

%\surroundwithmdframed[width=\columnwidth]{lstlisting}
\def\inputGnumericTable{}                                 %%
\lstset{
%language=C,
frame=single, 
breaklines=true,
columns=fullflexible
}

\begin{document}
%
\tikzstyle{block} = [rectangle, draw,
    text width=3em, text centered, minimum height=3em]
\tikzstyle{sum} = [draw, circle, node distance=3cm]
\tikzstyle{input} = [coordinate]
\tikzstyle{output} = [coordinate]
\tikzstyle{pinstyle} = [pin edge={to-,thin,black}]

\theoremstyle{definition}
\newtheorem{theorem}{Theorem}[section]
\newtheorem{problem}{Problem}
\newtheorem{proposition}{Proposition}[section]
\newtheorem{lemma}{Lemma}[section]
\newtheorem{corollary}[theorem]{Corollary}
\newtheorem{example}{Example}[section]
\newtheorem{definition}{Definition}[section]
%\newtheorem{algorithm}{Algorithm}[section]
%\newtheorem{cor}{Corollary}
\newcommand{\BEQA}{\begin{eqnarray}}
\newcommand{\EEQA}{\end{eqnarray}}
\newcommand{\define}{\stackrel{\triangle}{=}}

\bibliographystyle{IEEEtran}
%\bibliographystyle{ieeetr}

\providecommand{\nCr}[2]{\,^{#1}C_{#2}} % nCr
\providecommand{\nPr}[2]{\,^{#1}P_{#2}} % nPr
\providecommand{\mbf}{\mathbf}
\providecommand{\mtx}[1]{\mathbf{#1}}
\providecommand{\pr}[1]{\ensuremath{\Pr\left(#1\right)}}
\providecommand{\qfunc}[1]{\ensuremath{Q\left(#1\right)}}
\providecommand{\sbrak}[1]{\ensuremath{{}\left[#1\right]}}
\providecommand{\lsbrak}[1]{\ensuremath{{}\left[#1\right.}}
\providecommand{\rsbrak}[1]{\ensuremath{{}\left.#1\right]}}
\providecommand{\brak}[1]{\ensuremath{\left(#1\right)}}
\providecommand{\lbrak}[1]{\ensuremath{\left(#1\right.}}
\providecommand{\rbrak}[1]{\ensuremath{\left.#1\right)}}
\providecommand{\cbrak}[1]{\ensuremath{\left\{#1\right\}}}
\providecommand{\lcbrak}[1]{\ensuremath{\left\{#1\right.}}
\providecommand{\rcbrak}[1]{\ensuremath{\left.#1\right\}}}
\theoremstyle{remark}
\newtheorem{rem}{Remark}
\newcommand{\sgn}{\mathop{\mathrm{sgn}}}
\providecommand{\ztrans}{\overset{\mathcal{Z}}{ \rightleftharpoons}}
\providecommand{\abs}[1]{\left\vert#1\right\vert}
\providecommand{\res}[1]{\Res\displaylimits_{#1}} 
\providecommand{\norm}[1]{\left\Vert#1\right\Vert}
\providecommand{\mtx}[1]{\mathbf{#1}}
\providecommand{\pd}[2]{\ensuremath{\frac{\partial #1}{\partial #2}}}
\providecommand{\mean}[1]{E\left[ #1 \right]}
\providecommand{\fourier}{\overset{\mathcal{F}}{ \rightleftharpoons}}
\providecommand{\gauss}[2]{\mathcal{N}\ensuremath{\left(#1,#2\right)}}
%\providecommand{\hilbert}{\overset{\mathcal{H}}{ \rightleftharpoons}}
\providecommand{\system}{\overset{\mathcal{H}}{ \longleftrightarrow}}
	%\newcommand{\solution}[2]{\textbf{Solution:}{#1}}
\newcommand{\solution}{\noindent \textbf{Solution: }}
\newcommand{\myvec}[1]{\ensuremath{\begin{pmatrix}#1\end{pmatrix}}}
\providecommand{\dec}[2]{\ensuremath{\overset{#1}{\underset{#2}{\gtrless}}}}
\DeclarePairedDelimiter{\ceil}{\lceil}{\rceil}
%\numberwithin{equation}{section}
%\numberwithin{problem}{subsection}
%\numberwithin{definition}{subsection}
\makeatletter
\@addtoreset{figure}{section}
\makeatother

\let\StandardTheFigure\thefigure
%\renewcommand{\thefigure}{\theproblem.\arabic{figure}}
\renewcommand{\thefigure}{\thesection}


%\numberwithin{figure}{subsection}

%\numberwithin{equation}{subsection}
%\numberwithin{equation}{section}
%\numberwithin{equation}{problem}
%\numberwithin{problem}{subsection}
\numberwithin{problem}{section}
%%\numberwithin{definition}{subsection}
%\makeatletter
%\@addtoreset{figure}{problem}
%\makeatother
\makeatletter
\@addtoreset{table}{section}
\makeatother

\let\StandardTheFigure\thefigure
\let\StandardTheTable\thetable
\let\vec\mathbf
\numberwithin{equation}{section}

\vspace{3cm}


\title{%Convex Optimization in Python
	Random Numbers
}
%\title{
%	\logo{Matrix Analysis through Octave}{\begin{center}\includegraphics[scale=.24]{tlc}\end{center}}{}{HAMDSP}
%}

% paper title
% can use linebreaks \\ within to get better formatting as desired
%\title{Matrix Analysis through Octave}
%
%
% author names and IEEE memberships
% note positions of commas and nonbreaking spaces ( ~ ) LaTeX will not break
% a structure at a ~ so this keeps an author's name from being broken across
% two lines.
% use \thanks{} to gain access to the first footnote area
% a separate \thanks must be used for each paragraph as LaTeX2e's \thanks
% was not built to handle multiple paragraphs
%

\author{JARPULA BHANU PRASAD - AI21BTECH11015}
\maketitle

\tableofcontents

\bigskip

\renewcommand{\thefigure}{\theenumi}
\renewcommand{\thetable}{\theenumi}

\begin{abstract}
This manual provides a simple introduction to digital signal processing.
\end{abstract}
%%
\section{Software installation}

\begin{enumerate}[label=\thesection.\arabic*
,ref=\thesection.\theenumi]
\item Run the following commands.
\begin{lstlisting}
sudo apt-get update
sudo apt-get install libffi-dev libsndfile1 python3-scipy python3-numpy python3-matplotlib
sudo pip install cffi pysoundfile
\end{lstlisting}
\end{enumerate}

\section{Digital Filter}

\begin{enumerate}[label=\thesection.\arabic*
,ref=\thesection.\theenumi]
\item Download the sound file from
\begin{lstlisting}
wget https://github.com/jarpula-Bhanu/EE3900/blob/main/Ass/soundfiles/Sound_Noise.wav
\end{lstlisting}

\item You will find a spectogram at https://academo.org/demos/spectrum-analyzer. Upload the sound file that you downloaded in problem 2.1 in the spectrogram and play. Observe the spectogram. What do you find?
\solution There are a lot of yellow lines betweeen 440Hz to 5.1KHz. These represent the synthesizer key tones. Also, the key strokes are audible along with background noise.

\item Write the python code for removal of out of band nosie and execute the code.\label{2.3}\\
\solution Download and run the following code.
\begin{lstlisting}
wget https://github.com/jarpula-Bhanu/EE3900/blob/main/Ass/Codes/2.3_noise.py
\end{lstlisting}
run the above code using the command
\begin{lstlisting}
	python3 2.3_noise.py
\end{lstlisting}

\item The output of the python scripy in problem 2.3 is the audio file Sound\_With\_ReducedNoise.wav. Play the file in the spectogram in problem 2.2. What do you observe?\\
\solution The key strokes as well as background noise is subdued in the audio. Also the signal is blank for frequencies above 5.1KHz.
\end{enumerate}


\section{Difference Equation}

\begin{enumerate}[label=\thesection.\arabic*
,ref=\thesection.\theenumi]
\item Let 
\begin{align}
	x(n) = \cbrak{\underset{\uparrow}{1},2,3,4,2,1}
\end{align}
Sketch x(n).\\
\solution Download and run the following code.Below code plots fig\eqref{fig:3.1}
\begin{lstlisting}
wget https://github.com/jarpula-Bhanu/EE3900/blob/main/Ass/Codes/3.1.py
\end{lstlisting}
run the above code using the command
\begin{lstlisting}
	python3 3.1.py
\end{lstlisting}
\begin{figure}[h]
    \centering
    \includegraphics[width=\columnwidth]{./figs/3.1.png}
    \caption{Sketch of $x(n)$}
    \label{fig:3.1}
\end{figure}

\item Let
	\begin{align}\label{3.2}
		y(n) + \frac{1}{2}y(n-1) = x(n)+x(n-2),
		y(n)=0,n<0
	\end{align}
Sketech y(n).\\
\solution Download and run the following code.Below code plots fig\eqref{fig:3.2}
\begin{lstlisting}
wget https://github.com/jarpula-Bhanu/EE3900/blob/main/Ass/Codes/3.2.py
\end{lstlisting}
run the above code using the command
\begin{lstlisting}
	python3 3.2.py
\end{lstlisting}
\begin{figure}[h]
    \centering
    \includegraphics[width=\columnwidth]{./figs/3.2.png}
    \caption{Sketch of $x(n)$ and $y(n)$}
    \label{fig:3.2}
\end{figure}

\item Repeat the above exercise using C code.\\
\solution Download and run the following code.Below code plots fig\eqref{fig:3.3}
\begin{lstlisting}
wget https://github.com/jarpula-Bhanu/EE3900/blob/main/Ass/Codes/3.3.c
wget https://github.com/jarpula-Bhanu/EE3900/blob/main/Ass/Codes/3.3_plot.py
\end{lstlisting}
run the above code using the command
\begin{lstlisting}
	gcc 3.3.c -o 3.3.out
	./3.3.out
	python3 3.3_plot.py
\end{lstlisting}
\begin{figure}[h]
    \centering
    \includegraphics[width=\columnwidth]{./figs/3.3_plot.png}
    \caption{Sketch of $y(n)$}
    \label{fig:3.3}
\end{figure}


\end{enumerate}

\section{$Z$-transform}

\begin{enumerate}[label=\thesection.\arabic*
,ref=\thesection.\theenumi]
\item The Z-transform of $x(n)$ is defined as 
\begin{align}\label{4.1}
	X(z) = \mathcal{Z}\cbrak{x(n)} = \sum_{n=-\infty}^\infty x(n) z^{-n}
\end{align}
Show that
\begin{align}\label{4.2}
	\mathcal{Z}\cbrak{x(n-1)} = z^{-1}X(z)
\end{align}
and find
\begin{align}
	\mathcal{Z}\cbrak{x(n-k)}
\end{align}
\solution From \eqref{4.1}
\begin{align}
	\mathcal{Z}\cbrak{x(n-k)} &= \sum_{n=-\infty}^\infty x(n-1)z^{-n}\\
	&= \sum_{n=-\infty}^\infty x(n)z^{-n-1}= z^{-1}\sum_{n=-\infty}^\infty x(n)z^{-n}
\end{align}
resulting in \eqref{4.2}. Similarly, it can be shown that 
\begin{align}\label{4.6}
	\mathcal{Z}\cbrak{x(n-k)} = z^{-k}X(z)
\end{align}


\item Find
%
\begin{equation}\label{4.7}
H(z) = \frac{Y(z)}{X(z)}
\end{equation}
%
from  \eqref{3.2} assuming that the $Z$-transform is a linear operation.
\\
\solution  Applying \eqref{4.6} in \eqref{3.2},
\begin{align}
Y(z) + \frac{1}{2}z^{-1}Y(z) &= X(z)+z^{-2}X(z)
\\
\implies \frac{Y(z)}{X(z)} &= \frac{1 + z^{-2}}{1 + \frac{1}{2}z^{-1}}
\label{eq:freq_resp}
\end{align}
%
\item Find the Z transform of 
\begin{equation}
\delta(n)
=
\begin{cases}
1 & n = 0
\\
0 & \text{otherwise}
\end{cases}
\end{equation}
and show that the $Z$-transform of
\begin{equation}
\label{eq:unit_step}
u(n)
=
\begin{cases}
1 & n \ge 0
\\
0 & \text{otherwise}
\end{cases}
\end{equation}
is
\begin{equation}
U(z) = \frac{1}{1-z^{-1}}, \quad \abs{z} > 1
\end{equation}
\solution It is easy to show that
\begin{equation}
\delta(n) \ztrans 1
\end{equation}
and from \eqref{eq:unit_step},
\begin{align}
U(z) &= \sum _{n= 0}^{\infty}z^{-n}
\\
&=\frac{1}{1-z^{-1}}, \quad \abs{z} > 1
\end{align}
using the fomula for the sum of an infinite geometric progression.
%
\item Show that 
\begin{equation}
\label{eq:anun}
a^nu(n) \ztrans \frac{1}{1-az^{-1}} \quad \abs{z} > \abs{a}
\end{equation}
\solution \begin{align}
	a^nu(n) &\ztrans \sum_{n=0}^\infty (az^{-1})^n\\
	& =\frac{1}{1-az^{-1}} \quad \abs{z} > \abs{a}
\end{align}
%
\item 
Let
\begin{equation}
H\brak{e^{\j \omega}} = H\brak{z = e^{\j \omega}}.
\end{equation}
Plot $\abs{H\brak{e^{\j \omega}}}$.  Comment.  $H(e^{\j \omega})$ is
known as the {\em Discret Time Fourier Transform} (DTFT) of $x(n)$.
\\
\solution Download and run the following code. The following code plots Fig. \ref{fig:dtft}.
\begin{lstlisting}
wget https://github.com/jarpula-Bhanu/EE3900/blob/main/Ass/Codes/4.5.py
\end{lstlisting}
run the above code using the command 
\begin{lstlisting}
	python3 4.5.py
\end{lstlisting}
We observe that $\abs{H\brak{e^{\j \omega}}}$ is periodic with fundamental period $2\pi$.
\begin{figure}[!ht]
\centering
\includegraphics[width=\columnwidth]{./figs/4.5.png}
\caption{$\abs{H\brak{e^{\j\omega}}}$}
\label{fig:dtft}
\end{figure}

\end{enumerate}

\section{Impulse Response}

\begin{enumerate}[label=\thesection.\arabic*
,ref=\thesection.\theenumi]
\item \label{prob:impulse_resp}
Find an expression for $h(n)$ using $H(z)$, given that 
%in Problem \ref{eq:ztransab} and \eqref{eq:anun}, given that
\begin{equation}
\label{eq:impulse_resp}
h(n) \ztrans H(z)
\end{equation}
and there is a one to one relationship between $h(n)$ and $H(z)$. $h(n)$ is known as the {\em impulse response} of the
system defined by \eqref{3.2}.
\\
\solution From \eqref{eq:freq_resp},
\begin{align}
H(z) &= \frac{1}{1 + \frac{1}{2}z^{-1}} + \frac{ z^{-2}}{1 + \frac{1}{2}z^{-1}}
\\
\implies h(n) &= \brak{-\frac{1}{2}}^{n}u(n) + \brak{-\frac{1}{2}}^{n-2}u(n-2)
\end{align}
using \eqref{eq:anun} and \eqref{4.6}.
\item Sketch $h(n)$. Is it bounded? Convergent? 
\\
\solution Download and run the following code.The following code plots Fig. \ref{fig:hn}.
\begin{lstlisting}
wget https://github.com/jarpula-Bhanu/EE3900/blob/main/Ass/Codes/5.2.py
\end{lstlisting}
run the above code using the command.
\begin{lstlisting}
	python3 5.2.py
\end{lstlisting}
\begin{figure}[!ht]
\centering
\includegraphics[width=\columnwidth]{./figs/5.2.png}
\caption{$h(n)$ as the inverse of $H(z)$}
\label{fig:hn}
\end{figure}
$h(n)$ is bounded and convergent.
%
\item The system with $h(n)$ is defined to be stable if
\begin{equation}
\sum_{n=-\infty}^{\infty}h(n) < \infty
\end{equation}
Is the system defined by \eqref{3.2} stable for the impulse response in \eqref{eq:impulse_resp}?\\
%
\solution Note that
\begin{align}
	\sum_{n=-\infty}^\infty h(n) &= \sum_{n=-\infty}^\infty \brak{-\frac{1}{2}}^nu(n)+\brak{-\frac{1}{2}}^{n-2}u(n-2)\\
	&= 2\brak{\frac{1}{1+\frac{1}{2}}} = \frac{4}{3}
\end{align}
Thus, the given system is stable.

\item Compute and sketch $h(n)$ using 
\begin{equation}
\label{eq:iir_filter_h}
h(n) + \frac{1}{2}h(n-1) = \delta(n) + \delta(n-2), 
\end{equation}
%
This is the definition of $h(n)$.
\\
\solution The following code plots Fig. \ref{fig:hndef}. Note that this is the same as Fig. 
\ref{fig:hn}. 
%
\begin{lstlisting}
wget https://github.com/jarpula-Bhanu/EE3900/blob/main/Ass/Codes/5.4.py
\end{lstlisting}
run the above code using the command.
\begin{lstlisting}
	python3 5.4.py
\end{lstlisting}
\begin{figure}[!ht]
\centering
\includegraphics[width=\columnwidth]{./figs/5.4.png}
\caption{$h(n)$ from the definition}
\label{fig:hndef}
\end{figure}
%
\item Compute 
%
\begin{equation}
\label{eq:convolution}
y(n) = x(n)*h(n) = \sum_{n=-\infty}^{\infty}x(k)h(n-k)
\end{equation}
%
Comment. The operation in \eqref{eq:convolution} is known as
{\em convolution}.
%
\\
\solution The following code plots Fig. \ref{fig:ynconv}. Note that this is the same as 
$y(n)$ in  Fig. 
\ref{fig:3.2}. 
%
\begin{lstlisting}
wget https://github.com/jarpula-Bhanu/EE3900/blob/main/Ass/Codes/5.5.py
\end{lstlisting}
run the above code using the command.
\begin{lstlisting}
	python3 5.5.py
\end{lstlisting}
\begin{figure}[!ht]
\centering
\includegraphics[width=\columnwidth]{./figs/5.5.png}
\caption{$y(n)$ from the definition of convolution}
\label{fig:ynconv}
\end{figure}
\item Show that
\begin{equation}
y(n) =  \sum_{n=-\infty}^{\infty}x(n-k)h(k)
\end{equation}
\solution from \ref{eq:convolution}, we substitute $k := n-k$ to get
\begin{align}
	y(n) &= \sum_{k=-\infty}^\infty x(k)h(n-k)\\
	&= \sum_{n-k=-\infty}^\infty x(n-k)h(k)\\
	&= \sum_{k=-\infty}^\infty x(n-k)h(k)
\end{align}

\end{enumerate}

\section{DFT and FFT}

\begin{enumerate}[label=\thesection.\arabic*
,ref=\thesection.\theenumi]
\item
Compute
\begin{equation}
X(k) \define \sum _{n=0}^{N-1}x(n) e^{-\j2\pi kn/N}, \quad k = 0,1,\dots, N-1
\end{equation}
and $H(k)$ using $h(n)$.\\
\solution The following code plots Fig. \ref{fig:6.1}.
%
\begin{lstlisting}
wget https://github.com/jarpula-Bhanu/EE3900/blob/main/Ass/Codes/6.1.py
\end{lstlisting}
run the above code using the command.
\begin{lstlisting}
	python3 6.1.py
\end{lstlisting}
\begin{figure}[!ht]
\centering
\includegraphics[width=\columnwidth]{./figs/6.1.png}
\caption{Plots of the real parts of the DFT of $x(n)$ and $h(n)$}
\label{fig:6.1}
\end{figure}

\item Compute 
\begin{equation}\label{6.2}
Y(k) = X(k)H(k)
\end{equation}
\solution Download and run the following code.
%
\begin{lstlisting}
wget https://github.com/jarpula-Bhanu/EE3900/blob/main/Ass/Codes/6.2.py
\end{lstlisting}
run the above code using the command.
\begin{lstlisting}
	python3 6.2.py
\end{lstlisting}

\item Compute
\begin{equation} \label{6.3}
 y\brak{n}={\frac {1}{N}}\sum _{k=0}^{N-1}Y\brak{k}\cdot e^{\j 2\pi kn/N},\quad n = 0,1,\dots, N-1
\end{equation}
\\
\solution The following code plots Fig. \ref{fig:ynconv}. Note that this is the same as 
$y(n)$ in  Fig. 
\ref{fig:3.2}. 
%
\begin{lstlisting}
wget https://github.com/jarpula-Bhanu/EE3900/blob/main/Ass/Codes/6.3.py
\end{lstlisting}
run the above code using the command.
\begin{lstlisting}
	python3 6.3.py
\end{lstlisting}

\begin{figure}[!ht]
\centering
\includegraphics[width=\columnwidth]{./figs/6.3.png}
\caption{$y(n)$ from the DFT}
\label{fig:yndft}
\end{figure}

\item Repeat the previous exercise by computing $X(k), H(k)$ and $y(n)$ through FFT and 
 IFFT.
 \solution Download the code from
\begin{lstlisting}
wget https://github.com/jarpula-Bhanu/EE3900/blob/main/Ass/Codes/6.4.py
% \end{lstlisting}
and execute it using
\begin{lstlisting}
$ python3 6.4.py
\end{lstlisting}
Observe that Fig. \eqref{fig:y-n-fft} is the same as $y(n)$ in Fig. \eqref{fig:3.2}.
\begin{figure}
\centering
\includegraphics[width=\columnwidth]{figs/6.4.png}
\caption{$y(n)$ using FFT and IFFT}
\label{fig:y-n-fft}
\end{figure}
\item Wherever possible, express all the above equations as matrix equations.\\
\solution
We use the DFT Matrix, where $\omega = e^{-\frac{j2k\pi}{N}}$, which is given by
\begin{align}
	\mtx{W} = 
	\begin{pmatrix}
		\omega^0 & \omega^0 & \ldots & \omega^0 \\
		\omega^0 & \omega^1 & \ldots & \omega^{N - 1} \\
		\vdots & \vdots & \ddots & \vdots \\
		\omega^0 & \omega^{N - 1} & \ldots & \omega^{(N -1)(N - 1)}
	\end{pmatrix}
\end{align}
i.e. $W_{jk} = \omega^{jk}$, $0 \leq j, k < N$. Hence, we can write any DFT equation as
\begin{align}
	\mtx{X} = \mtx{W}\mtx{x} = \mtx{x}\mtx{W}
\end{align}
\noindent where
\begin{align}
	\mtx{x} = 
	\begin{pmatrix}
		x(0) \\ x(1) \\ \vdots \\ x(n - 1)
	\end{pmatrix}
\end{align}
\noindent Using \eqref{6.3}, the inverse Fourier Transform is given by
\begin{align}
	\mtx{x} = \mathcal{F}^{-1}\brak{\mtx{X}} = \mtx{W}^{-1}\mtx{X} &= \frac{1}{N}\mtx{W^{H}}\mtx{X} = \frac{1}{N}\mtx{X}\mtx{W^{H}} \\ 
	\implies \mtx{W}^{-1} &= \frac{1}{N}\mtx{W^{H}}
\end{align}
\noindent where $H$ denotes hermitian operator. We can rewrite \eqref{6.2} using the element-wise multiplication operator as
\begin{align}
	\mtx{Y} = \mtx{H}\cdot\mtx{X} = \brak{\mtx{W}\mtx{h}}\cdot\brak{\mtx{W}\mtx{x}}
\end{align}
\end{enumerate}

\section{Exercises}
Answer the following questions by looking at the python code in Problem \ref{2.3}.
\begin{enumerate}[label=\thesection.\arabic*]
\item
The command
\begin{lstlisting}
	output_signal = signal.lfilter(b, a, input_signal)
	\end{lstlisting}
in Problem \ref{2.3} is executed through the following difference equation
\begin{equation}
\label{eq:iir_filter_gen}
 \sum _{m=0}^{M}a\brak{m}y\brak{n-m}=\sum _{k=0}^{N}b\brak{k}x\brak{n-k}
\end{equation}
%
where the input signal is $x(n)$ and the output signal is $y(n)$ with initial values all 0. Replace
\textbf{signal.filtfilt} with your own routine and verify.\\
\solution  \solution Download the code from
\begin{lstlisting}
wget https://github.com/jarpula-Bhanu/EE3900/blob/main/Ass/Codes/7.1.py
\end{lstlisting}
and execute it using
\begin{lstlisting}
$ python3 7.1.py
\end{lstlisting}
%
\item Repeat all the exercises in the previous sections for the above $a$ and $b$.\\
\solution The filter frequency  response is plotted at 
\begin{lstlisting}
wget https://github.com/jarpula-Bhanu/EE3900/blob/main/Ass/Codes/7.2.1.py
\end{lstlisting}
The impulse response function is plotted at 
\begin{lstlisting}
wget https://github.com/jarpula-Bhanu/EE3900/blob/main/Ass/Codes/7.2.2.py
\end{lstlisting}
We see that $h(n)$ is bounded and convergent.Also,since 1 is not a pole of the transfer function, the system is stable.\\
run the above codes using
\begin{lstlisting}
	python3 7.2.1.py
	python3 7.2.2.py
\end{lstlisting}
\begin{figure}[!ht]
\centering
\includegraphics[width=\columnwidth]{./figs/7.2.1.png}
\caption{Filter frequency response}
\label{fig:7.2.1}
\includegraphics[width=\columnwidth]{./figs/7.2.2.png}
\caption{Plot of $h(n)$}
\label{fig:7.2.2}
\end{figure}
\item What is the sampling frequency of the input signal?
\\
\solution
Sampling frequency(fs)=44.1kHZ.
\item
What is type, order and  cutoff-frequency of the above butterworth filter
\\
\solution
The given butterworth filter is low pass with order=4 and cutoff-frequency=4kHz.
%
\item
Modifying the code with different input parameters and to get the best possible output.\\
%
\solution A better filtering was found on setting te order of the filter to be 7
\end{enumerate}

\end{document}