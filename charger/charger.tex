\documentclass[journal,12pt,twocolumn]{IEEEtran}
%
\usepackage{setspace}
\usepackage{gensymb}
\usepackage{xcolor}
\usepackage{caption}
%\usepackage{subcaption}
%\doublespacing
\singlespacing

%\usepackage{graphicx}
%\usepackage{amssymb}
%\usepackage{relsize}
\usepackage[cmex10]{amsmath}
\usepackage{mathtools}
%\usepackage{amsthm}
%\interdisplaylinepenalty=2500
%\savesymbol{iint}
%\usepackage{txfonts}
%\restoresymbol{TXF}{iint}
%\usepackage{wasysym}
\usepackage{hyperref}
\usepackage{amsthm}
\usepackage{mathrsfs}
\usepackage{txfonts}
\usepackage{stfloats}
\usepackage{cite}
\usepackage{cases}
\usepackage{subfig}
%\usepackage{xtab}
\usepackage{longtable}
\usepackage{multirow}
%\usepackage{algorithm}
%\usepackage{algpseudocode}
%\usepackage{enumerate}
\usepackage{enumitem}
\usepackage{mathtools}
\usepackage{circuitikz}
%\usepackage{iithtlc}
%\usepackage[framemethod=tikz]{mdframed}
\usepackage{listings}
\let\vec\mathbf


%\usepackage{stmaryrd}


%\usepackage{wasysym}
%\newcounter{MYtempeqncnt}
\DeclareMathOperator*{\Res}{Res}
%\renewcommand{\baselinestretch}{2}
\renewcommand\thesection{\arabic{section}}
\renewcommand\thesubsection{\thesection.\arabic{subsection}}
\renewcommand\thesubsubsection{\thesubsection.\arabic{subsubsection}}

\renewcommand\thesectiondis{\arabic{section}}
\renewcommand\thesubsectiondis{\thesectiondis.\arabic{subsection}}
\renewcommand\thesubsubsectiondis{\thesubsectiondis.\arabic{subsubsection}}

%\renewcommand{\labelenumi}{\textbf{\theenumi}}
%\renewcommand{\theenumi}{P.\arabic{enumi}}

% correct bad hyphenation here
\hyphenation{op-tical net-works semi-conduc-tor}

\lstset{
language=Python,
frame=single,
breaklines=true,
columns=fullflexible
}



\begin{document}
%

\theoremstyle{definition}
\newtheorem{theorem}{Theorem}[section]
\newtheorem{problem}{Problem}
\newtheorem{proposition}{Proposition}[section]
\newtheorem{lemma}{Lemma}[section]
\newtheorem{corollary}[theorem]{Corollary}
\newtheorem{example}{Example}[section]
\newtheorem{definition}{Definition}[section]
%\newtheorem{algorithm}{Algorithm}[section]
%\newtheorem{cor}{Corollary}
\newcommand{\BEQA}{\begin{eqnarray}}
\newcommand{\EEQA}{\end{eqnarray}}
\newcommand{\define}{\stackrel{\triangle}{=}}
\newcommand{\myvec}[1]{\ensuremath{\begin{pmatrix}#1\end{pmatrix}}}
\newcommand{\mydet}[1]{\ensuremath{\begin{vmatrix}#1\end{vmatrix}}}

\bibliographystyle{IEEEtran}
%\bibliographystyle{ieeetr}

\providecommand{\nCr}[2]{\,^{#1}C_{#2}} % nCr
\providecommand{\nPr}[2]{\,^{#1}P_{#2}} % nPr
\providecommand{\mbf}{\mathbf}
\providecommand{\pr}[1]{\ensuremath{\Pr\left(#1\right)}}
\providecommand{\qfunc}[1]{\ensuremath{Q\left(#1\right)}}
\providecommand{\sbrak}[1]{\ensuremath{{}\left[#1\right]}}
\providecommand{\lsbrak}[1]{\ensuremath{{}\left[#1\right.}}
\providecommand{\rsbrak}[1]{\ensuremath{{}\left.#1\right]}}
\providecommand{\brak}[1]{\ensuremath{\left(#1\right)}}
\providecommand{\lbrak}[1]{\ensuremath{\left(#1\right.}}
\providecommand{\rbrak}[1]{\ensuremath{\left.#1\right)}}
\providecommand{\cbrak}[1]{\ensuremath{\left\{#1\right\}}}
\providecommand{\lcbrak}[1]{\ensuremath{\left\{#1\right.}}
\providecommand{\rcbrak}[1]{\ensuremath{\left.#1\right\}}}
\theoremstyle{remark}
\newtheorem{rem}{Remark}
\newcommand{\sgn}{\mathop{\mathrm{sgn}}}
\providecommand{\abs}[1]{\left\vert#1\right\vert}
\providecommand{\res}[1]{\Res\displaylimits_{#1}}
\providecommand{\norm}[1]{\lVert#1\rVert}
\providecommand{\mtx}[1]{\mathbf{#1}}
\providecommand{\mean}[1]{E\left[ #1 \right]}
\providecommand{\rect}[1]{\text{rect}\ensuremath{\left(#1\right)}}
\providecommand{\sinc}[1]{\text{sinc}\ensuremath{\left(#1\right)}}
\providecommand{\fourier}{\overset{\mathcal{F}}{ \rightleftharpoons}}
\providecommand{\ztrans}{\overset{\mathcal{Z}}{ \rightleftharpoons}}

%\providecommand{\hilbert}{\overset{\mathcal{H}}{ \rightleftharpoons}}
\providecommand{\system}{\overset{\mathcal{H}}{ \longleftrightarrow}}
	%\newcommand{\solution}[2]{\textbf{Solution:}{#1}}
\newcommand{\solution}{\noindent \textbf{Solution: }}
\providecommand{\dec}[2]{\ensuremath{\overset{#1}{\underset{#2}{\gtrless}}}}
\numberwithin{equation}{section}
%\numberwithin{equation}{subsection}
%\numberwithin{problem}{subsection}
%\numberwithin{definition}{subsection}
\makeatletter
\@addtoreset{figure}{problem}
\makeatother

\let\StandardTheFigure\thefigure
%\renewcommand{\thefigure}{\theproblem.\arabic{figure}}
\renewcommand{\thefigure}{\theproblem}


%\numberwithin{figure}{subsection}

\def\putbox#1#2#3{\makebox[0in][l]{\makebox[#1][l]{}\raisebox{\baselineskip}[0in][0in]{\raisebox{#2}[0in][0in]{#3}}}}
     \def\rightbox#1{\makebox[0in][r]{#1}}
     \def\centbox#1{\makebox[0in]{#1}}
     \def\topbox#1{\raisebox{-\baselineskip}[0in][0in]{#1}}
     \def\midbox#1{\raisebox{-0.5\baselineskip}[0in][0in]{#1}}

\vspace{3cm}

\title{
%\logo{
%}
Fourier Series
%	\logo{Octave for Math Computing }
}
%\title{
%	\logo{Matrix Analysis through Octave}{\begin{center}\includegraphics[scale=.24]{tlc}\end{center}}{}{HAMDSP}
%}


% paper title
% can use linebreaks \\ within to get better formatting as desired
%\title{Matrix Analysis through Octave}
%
%
% author names and IEEE memberships
% note positions of commas and nonbreaking spaces ( ~ ) LaTeX will not break
% a structure at a ~ so this keeps an author's name from being broken across
% two lines.
% use \thanks{} to gain access to the first footnote area
% a separate \thanks must be used for each paragraph as LaTeX2e's \thanks
% was not built to handle multiple paragraphs
%

\author{Jarpula Bhanu Prasad - AI21BTECH11015}%<-this  stops a space
%\thanks{*The author is with the Department
%of Electrical Engineering, Indian Institute of Technology, Hyderabad
%502285 India e-mail:  gadepall@iith.ac.in.  All content in the manuscript is
%released under GNU GPL.  Free to use for anything. }% <-this % stops a space
%\thanks{J. Doe and J. Doe are with Anonymous University.}% <-this % stops a space
%\thanks{Manuscript received April 19, 2005; revised January 11, 2007.}}
%}
% note the % following the last \IEEEmembership and also \thanks -
% these prevent an unwanted space from occurring between the last author name
% and the end of the author line. i.e., if you had this:
%
% \author{....lastname \thanks{...} \thanks{...} }
%                     ^------------^------------^----Do not want these spaces!
%
% a space would be appended to the last name and could cause every name on that
% line to be shifted left slightly. This is one of those "LaTeX things". For
% instance, "\textbf{A} \textbf{B}" will typeset as "A B" not "AB". To get
% "AB" then you have to do: "\textbf{A}\textbf{B}"
% \thanks is no different in this regard, so shield the last } of each \thanks
% that ends a line with a % and do not let a space in before the next \thanks.
% Spaces after \IEEEmembership other than the last one are OK (and needed) as
% you are supposed to have spaces between the names. For what it is worth,
% this is a minor point as most people would not even notice if the said evil
% space somehow managed to creep in.



% The paper headers
%\markboth{Journal of \LaTeX\ Class Files,~Vol.~6, No.~1, January~2007}%
%{Shell \MakeLowercase{\textit{et al.}}: Bare Demo of IEEEtran.cls for Journals}
% The only time the second header will appear is for the odd numbered pages
% after the title page when using the twoside option.
%
% *** Note that you probably will NOT want to include the author's ***
% *** name in the headers of peer review papers.                   ***
% You can use \ifCLASSOPTIONpeerreview for conditional compilation here if
% you desire.




% If you want to put a publisher's ID mark on the page you can do it like
% this:
%\IEEEpubid{0000--0000/00\$00.00~\copyright~2007 IEEE}
% Remember, if you use this you must call \IEEEpubidadjcol in the second
% column for its text to clear the IEEEpubid mark.



% make the title area
\maketitle

%\newpage

\tableofcontents

%\renewcommand{\thefigure}{\thesection.\theenumi}
%\renewcommand{\thetable}{\thesection.\theenumi}

\renewcommand{\thefigure}{\theenumi}
\renewcommand{\thetable}{\theenumi}

%\renewcommand{\theequation}{\thesection}


\bigskip

\begin{abstract}
This manual provides a simple introduction to Fourier Series
\end{abstract}
\section{Periodic Function}
Let
\begin{align}
	x(t) &= A_0\abs{\sin\brak{2\pi f_0 t}}
	\label{eq:10-orig-diff-def}
\end{align}
\begin{enumerate}[label=\thesection.\arabic*
,ref=\thesection.\theenumi]
\item Plot $x(t)$.\\
\solution The following code will plot the graph in fig \eqref{plot}
\begin{lstlisting}
wget https://github.com/jarpula-Bhanu/EE3900/blob/main/charger/codes/1.1.py
\end{lstlisting}
run the above code using the command
\begin{lstlisting}
python3 1.1.py
\end{lstlisting}
\begin{figure}[!ht]
	\includegraphics[width=\columnwidth]{./figs/1.1.png}
	\caption{$x(t) = A_0\abs{\sin\brak{2\pi f_0 t}}$}
	\label{plot}
\end{figure}
\item Show that $x(t)$ is periodic and find its period.\\
\solution From  fig \eqref{plot}, we see that  $x(t)$ is periodic. Further,
\begin{align}
	\text{period of } \sin(at) \quad \text{given by } \frac{2\pi}{a}
\end{align}
Now, period of $x(t)$ is 
\begin{align}
	A_0\abs{\sin\brak{2\pi f_0 t}} &\implies \frac{\pi}{2\pi f_0}\\
	&\implies \frac{1}{2f_0}
\end{align}
Verification
\begin{align}
	x\brak{t+\frac{1}{2f_0}} &= A_0\abs{\sin\brak{2\pi f_0 \brak{t+\frac{1}{2f_0}}}}\\
	&= A_0\abs{\sin\brak{2\pi f_0 t+ \pi}}\\
	&= A_0\abs{-\sin\brak{2\pi f_0 t}}\\
	&= A_0\abs{\sin\brak{2\pi f_0 t}}
\end{align}
Hence the period of $x(t)$ is $\frac{1}{2f_0}$.
\end{enumerate}
\section{Fourier Series}
Consider $A_0 =12$ and $f_0 = 50$ for all numerical calculations.
\begin{enumerate}[label=\thesection.\arabic*,ref=\thesection.\theenumi]
\item If
%\cite{proakis_dsp}
\begin{align}
	x(t) = \sum_{k = -\infty}^{\infty}c_ke^{\j2\pi kf_0 t}
\label{eq:one-Z-complex}
\end{align}
show that
\begin{align}
	c_k = f_0\int_{-\frac{1}{2f_0}}^{\frac{1}{2f_0}}x(t)e^{-\j2\pi kf_0 t}\, dt
\label{eq:one-Z}
\end{align}
\solution We have for some $n \in \mathbb{Z}$,
\begin{align}
    x(t)e^{-\j2\pi nf_0t} = \sum_{k = -\infty}^{\infty}c_ke^{\j2\pi (k - n)f_0 t}
\end{align}
But we know from the periodicity of $e^{\j2\pi kf_0t}$,
\begin{align}
    \int_{-\frac{1}{2f_0}}^{\frac{1}{2f_0}}e^{\j2\pi kf_0 t}\, dt = 
    \frac{1}{f_0}\delta\brak{k} 
\end{align}
Thus,
\begin{align}
    \int_{-\frac{1}{2f_0}}^{\frac{1}{2f_0}}x(t)e^{-\j2\pi nf_0 t}\, dt = 
    \frac{c_n}{f_0} \\
    \implies c_n = f_0\int_{-\frac{1}{2f_0}}^{\frac{1}{2f_0}}x(t)e^{-\j2\pi nf_0 t}\, dt 
\end{align}
	\item Find $c_k$ for
	\eqref{eq:10-orig-diff-def}\\
	\solution Using \eqref{eq:one-Z},
	\begin{align}
		c_n &= f_0\int_{-\frac{1}{2f_0}}^{\frac{1}{2f_0}}A_0\abs{\sin\brak{2\pi f_0t}}
		e^{-\j2\pi nf_0t}\, dt \\
			&= f_0\int_{-\frac{1}{2f_0}}^{\frac{1}{2f_0}}A_0\abs{\sin\brak{2\pi f_0t}}
		\cos\brak{2\pi nf_0t}\, dt \nonumber \\
			&+ \j f_0\int_{-\frac{1}{2f_0}}^{\frac{1}{2f_0}}A_0
			\abs{\sin\brak{2\pi f_0t}}\sin\brak{2\pi nf_0t}\, dt \\
			&= 2f_0\int_{0}^{\frac{1}{2f_0}}A_0\sin\brak{2\pi f_0t}\cos\brak{2\pi nf_0t}\, dt \\
			&= f_0A_0\int_{0}^{\frac{1}{2f_0}}\brak{\sin\brak{2\pi\brak{n+1}f_0t}}\, dt \nonumber \\ 
			&- f_0A_0\int_{0}^{\frac{1}{2f_0}}\brak{\sin\brak{2\pi\brak{n-1}f_0t}}\, dt \\ 
			&= A_0\frac{1+\brak{-1}^n}{2\pi}\brak{\frac{1}{n+1} - \frac{1}{n-1}} \\
			&= 
			\begin{cases}
				\frac{2A_0}{\pi\brak{1-n^2}} & n\ \text{even} \\
				0 & n\ \text{odd}
			\end{cases}
	\end{align}

	\item Verify 
	\eqref{eq:one-Z-complex}
	using python.\\
	\solution The following code will plot the graph in fig \eqref{plot2}
	\begin{lstlisting}
wget https://github.com/jarpula-Bhanu/EE3900/blob/main/charger/codes/2.3.py
	\end{lstlisting}
	run the above code using the command
	\begin{lstlisting}
	python3 2.3.py
	\end{lstlisting}
	\begin{figure}[!ht]
		\includegraphics[width=\columnwidth]{./figs/2.3.png}
		\caption{$x(t) = \sum_{k = -\infty}^{\infty}c_ke^{\j2\pi kf_0 t}$}
		\label{plot2}
	\end{figure}
	\item Show that 
	\begin{align}
		x(t) = \sum_{k = 0}^{\infty}\brak{a_k\cos{\j2\pi kf_0 t}+b_k\sin{\j2\pi kf_0 t}}
		\label{eq:one-Z-real}
	\end{align}
	and obtain the formulae for $a_k$ and $b_k$.\\
	\solution From \eqref{eq:one-Z-complex},
	\begin{align}
		x(t) &= \sum_{k = -\infty}^{\infty}c_ke^{\j2\pi kf_0 t} \\
		&= c_0 + \sum_{k = 1}^{\infty}c_ke^{\j2\pi kf_0t} + c_{-k}e^{-\j2\pi kf_0t} \\
		&= c_0 + \sum_{k = 1}^{\infty}\brak{c_k + c_{-k}}\cos\brak{2\pi kf_0t}  \nonumber \\
		&+ \sum_{k = 0}^{\infty}\brak{c_k - c_{-k}}\sin\brak{2\pi kf_0t}
	\end{align}
	Hence, for $k \ge 0$,
	\begin{align}
		a_k &= 
		\begin{cases}
			c_0 & k = 0 \\
			c_k + c_{-k} & k > 0
		\end{cases} \\
		b_k &= c_k - c_{-k}
		\label{eq:akbk}
	\end{align}
	\item Find $a_k$ and $b_k$ for 
	\eqref{eq:10-orig-diff-def}\\
	\solution From \eqref{eq:one-Z-complex}, we see that since $x(t)$ is even,
	\begin{align}
		x(-t) &= \sum_{k = -\infty}^{\infty}c_ke^{-\j2\pi kf_0 t} \\
		&= \sum_{k = -\infty}^{\infty}c_{-k}e^{\j2\pi kf_0t} \label{eq:sub} \\
		&= \sum_{k = -\infty}^{\infty}c_ke^{\j2\pi kf_0 t}
	\end{align}
	where we substitute $k \mapsto -k$ in \eqref{eq:sub}. Hence, we see that 
	$c_k = c_{-k}$. So, from \eqref{eq:akbk} and for $k \ge 0$,
	\begin{align}
		a_k &= 
		\begin{cases}
			\frac{2A_0}{\pi} & k = 0 \\
			\frac{4A_0}{\pi\brak{1 - k^2}} & k > 0,\ k\ \text{even} \\
			0 & \text{otherwise}
		\end{cases} \\
		b_k &= 0
		\label{eq:akbk-even}
	\end{align}
	\item Verify
	\eqref{eq:one-Z-real}
	using python.\\
	\solution The following code will plot the graph in fig \eqref{plot3}
	\begin{lstlisting}
wget https://github.com/jarpula-Bhanu/EE3900/blob/main/charger/codes/2.6.py
	\end{lstlisting}
	run the above code using the command
	\begin{lstlisting}
	python3 2.6.py
	\end{lstlisting}
	\begin{figure}[!ht]
		\includegraphics[width=\columnwidth]{./figs/2.6.png}
		\caption{$x(t) = \sum_{k = 0}^{\infty}\brak{a_k\cos{\j2\pi kf_0 t}+b_k\sin{\j2\pi kf_0 t}}$}
		\label{plot3}
	\end{figure}
	
\end{enumerate}

\section{Fourier Transform}
\begin{enumerate}[label=\thesection.\arabic*
	,ref=\thesection.\theenumi]

	\item 
\begin{align}
	\delta(t)&=0, \quad t\neq 0 \\
	\int_{-\infty}^{\infty}\delta(t) \, dt&= 1
\end{align}
\item The Fourier Transform of $g(t)$ is
\begin{align}
G(f)=\int_{-\infty}^{\infty}g(t)e^{-j2\pi ft}\,dt
\label{eq:fourier}
\end{align}
\item Show that 
\begin{align}
    g(t-t_0)&\system{F}G(f)e^{-j2\pi ft_0}
    \label{eq:t-shift}
\end{align}
\solution We write, substituting $u := t-t_0$,
\begin{align}
    g(t-t_0)&\system{F}\int_{-\infty}^{\infty}
            g(t-t_0)e^{-\j2\pi ft}\,dt \\
            &=\int_{-\infty}^{\infty}
            g(u)e^{-\j2\pi f(u + t_0)}\,du \\
            &=G(f)e^{-j2\pi ft_0}
\end{align}
where the last equality follows from \eqref{eq:fourier}.
\item Show that
\begin{align} \label{duality}
    G(t)&\system{F}g(-f)
\end{align}
\solution Using the definition of the Inverse Fourier Transform,
\begin{align}
    g(t)=\int_{-\infty}^{\infty}G(f)e^{\j2\pi ft}\,df
    \label{eq:duality}
\end{align}
Hence, setting $t := -f$ and $f := t$, which implies $df = dt$,
\begin{align}
    g(-f)&=\int_{-\infty}^{\infty}G(t)e^{-\j2\pi ft}\,dt \\
    \implies G(t)&\system{F}g(-f)
\end{align}
	 \item $\delta(t)\system{F}?$\\
	 \solution By applying the defination fo fourier transformation for $\delta(t-t_0)$ \{$t_0$ be time shifting\}.
	 \begin{align}
		\mathcal{F}\cbrak{\delta(t-t_0)}(f) = \mathcal{F}(t)= \int_{-\infty}^{\infty} \delta(t-t_0)e^{-j2\pi ft}dt.
	 \end{align}
	 By applying the time shifting property of impulse we get
	 \begin{align}
		\mathcal{F}(f) &= e^{-j2\pi ft_0}\\
		i.e., \quad \delta(t-t_0) &\system{F} e^{-j2\pi ft_0}
	 \end{align}
	 Now, substitute $t_0 = 0$ we get 
	 \begin{align}
		\delta(t) \system{F} 1
	 \end{align}
	 \item $e^{-j2\pi f_0t}\system{F}?$\\
	 \solution Applying the defination of inverse fourier transformation.
	 \begin{align}
		\mathcal{F}^{-1}\cbrak{\delta(f+f_0)}(t) = f(t)= \int_{-\infty}^{\infty}\delta(f+f_0)e^{j2\pi ft}df
	 \end{align}
	 By applying the shifting property of impulse 
	 \begin{align}
		f(t) &= e^{-j2\pi f_0t}\\
		i.e., \quad e^{-j2\pi f_0t} &\system{F} \delta(f+f_0)
	 \end{align}
	 \item $\cos(2\pi f_0t)\system{F}?$\\
	 \solution \begin{align}
		\cos(2\pi f_0t) &= \frac{e^{j2\pi f_0t}+e^{-j2\pi f_0t}}{2}\\
		\mathcal{F}\sbrak{\cos(2\pi f_0t)} &= \int_{-\infty}^\infty \cos(2\pi f_0t) e^{-j2\pi ft}dt\\
		&= \int_{-\infty}^\infty \frac{e^{j2\pi f_0t}+e^{-j2\pi f_0t}}{2} e^{-j2\pi ft}dt\\
		&= \frac{1}{2}\Bigg[\int_{-\infty}^\infty e^{j2\pi f_0te^{-j2\pi ft}dt}\nonumber \\&\qquad+ \int_{-\infty}^\infty e^{-j2\pi f_0t}e^{-j2\pi ft}dt\Bigg]\\
		&= \frac{1}{2}\sbrak{\delta(f-f_0)+\delta(f+f_0)}\\
		\therefore \cos(2\pi f_0t)&\system{F}\frac{1}{2}\sbrak{\delta(f-f_0)+\delta(f+f_0)}
	 \end{align}
	 \item Find the Fourier Transform of $x(t)$ and plot it.  Verify using python.\\
	 \solution \begin{align}
		x(t) &= A_0 \abs{\sin(2\pi f_0 t)}\\
		X(f)&= \int_{-\infty}^\infty x(t)e^{-j2\pi ft}dt\\
		&= \int_{-\infty}^\infty A_0\abs{\sin(2\pi f_0t)}e^{-j2\pi ft}dt\\
		&= A_0 \int_0^\infty \sin(2\pi f_0t)e^{-j2\pi ft}dt + K\\
		&= \frac{A_0}{2j}\int_0^\infty( e^{j2\pi f_0t}+e^{-j2\pi f_0t}) e^{-j2\pi ft} dt + K\\
		&= \frac{A_0}{2j}\int_0^\infty( e^{j2\pi (f_0-f)t}+e^{-j2\pi (f_0-f)t}) dt + K\\
		&= \frac{A_0}{2j} \sbrak{0-\frac{1}{j2\pi (f_0-f)}}-\frac{A_0}{2j} \sbrak{0-\frac{1}{j2\pi (f_0+f)}}\\
		&= \frac{A_0}{2\pi (f_0^2-f^2)}+k
	 \end{align}
	 By symmetry we get 
	 \begin{align}
		k &= \frac{A_0}{2\pi (f_0^2-f^2)}\\
		\therefore \quad X(f)&= \frac{A_0}{\pi (f_0^2-f^2)}
	 \end{align}
	 The following code will plot the graph in fig \eqref{plot6}
	\begin{lstlisting}
wget https://github.com/jarpula-Bhanu/EE3900/blob/main/charger/codes/3.8.py
	\end{lstlisting}
	run the above code using the command
	\begin{lstlisting}
	python3 3.8.py
	\end{lstlisting}
	\begin{figure}[!ht]
		\includegraphics[width=\columnwidth]{./figs/3.8.png}
		\caption{$\mathcal{F}[x(t)]$}
		\label{plot6}
	\end{figure}
	 \item Show that 
	 \begin{align}
		 \rect{t} \system{F} \sinc{t}
	 \end{align}
	 Verify using python.\\
	 \solution \begin{align}
		\rect{t} &= \begin{cases}
			1 \quad \abs{t}<T\\
			0 \quad \text{otherwise}
		\end{cases}\\
		\mathcal{F}\sbrak{\rect{t}}&= \int_{-\infty}^\infty \rect{t}e^{-j2\pi ft}dt\\
		&= \int_{-T}^T e^{-j2\pi ft}dt\\
		&=\frac{1}{-j2\pi f}\sbrak{e^{-j2\pi fT}-e^{j2\pi fT}}\\
		&=\frac{1}{\pi f}\sbrak{ \frac{e^{j2\pi fT}-e^{-j2\pi fT}}{2j}}\\
		&=\frac{\sin{\pi f}}{\pi f}\\
		&= \sinc{f}
	 \end{align}
	 The following code will plot the graph in fig \eqref{plot4}
	\begin{lstlisting}
wget https://github.com/jarpula-Bhanu/EE3900/blob/main/charger/codes/3.9.py
	\end{lstlisting}
	run the above code using the command
	\begin{lstlisting}
	python3 3.9.py
	\end{lstlisting}
	\begin{figure}[!ht]
		\includegraphics[width=\columnwidth]{./figs/3.9.png}
		\caption{$\mathcal{F}[\rect{t}]$}
		\label{plot4}
	\end{figure}
	 \item $\sinc{t}\system{F} ?$.  Verify using python.\\
	 \solution from \eqref{duality} we can say
	 \begin{align}
		\sinc{t} &\system{F} \rect{-f}\\
		& \qquad \rect{f}
	 \end{align}
	 The following code will plot the graph in fig \eqref{plot5}
	 \begin{lstlisting}
 wget https://github.com/jarpula-Bhanu/EE3900/blob/main/charger/codes/3.10.py
	 \end{lstlisting}
	 run the above code using the command
	 \begin{lstlisting}
	 python3 3.10.py
	 \end{lstlisting}
	 \begin{figure}[!ht]
		 \includegraphics[width=\columnwidth]{./figs/3.10.png}
		 \caption{$\mathcal{F}[\sinc{t}]$}
		 \label{plot5}
	 \end{figure}
	\end{enumerate}

	\section{Filter}

\begin{enumerate}[label=\thesection.\arabic*
	,ref=\thesection.\theenumi]
	\item Find $H(f)$ which transforms $x(t)$ to DC 5V.\\
\solution The function $H(f)$ is a low pass filter which filters out
even harmonics and leaves the zero frequency component behind.
The rectangular function represents an ideal low pass filter. 
Suppose the cutoff frequency is $f_c = 50$ Hz, then
\begin{align}
    H(f) = \rect{\frac{f}{2f_c}} =
    \begin{cases}
        1 & \abs{f} < f_c \\
        0 & \textrm{otherwise}
    \end{cases}
    \label{eq:Hf}
\end{align}
Multiplying by a scaling factor to get DC 5V,
\begin{align}
    H(f) = \frac{\pi V_0}{2A_0}\rect{\brak{\frac{f}{2f_c}}}
\end{align}
where $V_0 = 5$ V.
\item Find $h(t)$.\\
\solution Suppose $g(t)\system{F}G(f)$. Then, for some
nonzero $a \in \mathbb{R}$
\begin{align}
    g(at)&\system{F}\int_{-\infty}^{\infty}g(at)e^{-\j2\pi ft}\, dt \\
         &=\frac{1}{a}\int_{-\infty}^{\infty}g(u)e^{\brak{-\j2\pi \frac{f}{a}t}}\, dt \\
         &=\frac{1}{a}G\brak{\frac{f}{a}}
         \label{eq:t-scaling}
\end{align}
where we have substituted $u := at$. Using 
\eqref{eq:t-scaling} of the Fourier Transform in \eqref{eq:Hf},
\begin{align}
    h(t) = \frac{2\pi V_0}{A_0}f_c \text{sinc} \brak{2f_ct}
\end{align}
	\item Verify your result using  through convolution.\\
	\solution
	The following code will plot the graph in fig \eqref{plot7}
	\begin{lstlisting}
wget https://github.com/jarpula-Bhanu/EE3900/blob/main/charger/codes/4.3.py
	\end{lstlisting}
	run the above code using the command
	\begin{lstlisting}
	python3 4.3.py
	\end{lstlisting}
	\begin{figure}[!ht]
		\includegraphics[width=\columnwidth]{./figs/4.3.png}
		\caption{convolution of two signals}
		\label{plot7}
	\end{figure}
	\end{enumerate}

\section{Filter Design}
\begin{enumerate}[label=\thesection.\arabic*
,ref=\thesection.\theenumi]
\item Design a Butterworth filter for $H(f)$.
\solution The Butterworth filter has an amplitude response
given by
\begin{align}
    \abs{H\brak{f}}^2 = \frac{1}{\brak{1 + \brak{\frac{f}{f_c}}^{2n}}}
\end{align}
where $n$ is the order of the filter and $f_c$ is the cutoff
frequency. The attenuation at frequency $f$ is given by 
\begin{align}
    A &= -10\log_{10}\abs{H\brak{f}}^2 \\
      &= -20\log_{10}\abs{H\brak{f}}
    \label{eq:loss}
\end{align}
We consider the following design parameters for our
lowpass analog Butterworth filter:
\begin{enumerate}
    \item Passband edge, $f_p = 50$ Hz
    \item Stopband edge, $f_s = 100$ Hz
    \item Passband attenuation, $A_p = -1$ dB
    \item Stopband attenuation, $A_s = -20$ dB
\end{enumerate}
We are required to find a desriable order $n$ and cutoff
frequency $f_c$ for the filter. From \eqref{eq:loss},
\begin{align}
    A_p &= -10\log_{10}\sbrak{1 + \brak{\frac{f_p}{f_c}}^{2n}} \\
    A_s &= -10\log_{10}\sbrak{1 + \brak{\frac{f_s}{f_c}}^{2n}}
\end{align}
Thus,
\begin{align}
    \brak{\frac{f_p}{f_c}}^{2n} = 10^{-\frac{A_p}{10}} - 1 \label{eq:fc1} \\
    \brak{\frac{f_s}{f_c}}^{2n} = 10^{-\frac{A_s}{10}} - 1 \label{eq:fc2}
\end{align}
Therefore, on dividing the above equations and solving for $n$,
\begin{align}
    n = \frac{\log\brak{10^{-\frac{A_s}{10}} - 1} - 
    \log\brak{10^{-\frac{A_p}{10}} - 1}}{2\brak{\log{f_s} - \log{f_p}}}
\end{align}
In this case, making appropriate susbstitutions gives $n = 4.29$.
Hence, we take $n = 5$. Solving for $f_c$ in \eqref{eq:fc1} and
\eqref{eq:fc2},
\begin{align}
    f_{c1} = f_p\sbrak{10^{-\frac{A_p}{10}} - 1}^{-\frac{1}{2n}} =  {57.23}{  \text{Hz}} \\
    f_{c2} = f_s\sbrak{10^{-\frac{A_s}{10}} - 1}^{-\frac{1}{2n}} =  {63.16}{  \text{Hz}}
\end{align}
Hence, we take $f_c = \sqrt{f_{c1}f_{c2}} =  {60}{  \text{Hz}}$ approximately.
\item Design a Chebyshev filter for $H(f)$.
\solution The Chebyshev filter has an amplitude response
given by
\begin{align}
    \abs{H\brak{f}}^2 = \frac{1}{\brak{1 + \epsilon^2C_n^2\brak{\frac{f}{f_c}}}}
\end{align}
where 
\begin{enumerate}
    \item $n$ is the order of the filter
    \item $\epsilon$ is the ripple
    \item $f_c$ is the cutoff frequency 
    \item $C_n = \cosh^{-1}\brak{n\cosh{x}}$ denotes 
    the n\textsuperscript{th} order Chebyshev polynomial,
    given by
    \begin{align}
        c_n(x) =
        \begin{cases}
            \cos\brak{n\cos^{-1}x} & \abs{x} \le 1 \\
            \cosh\brak{n\cosh^{-1}x} & \textrm{otherwise}
        \end{cases}
        \label{eq:chebypol}
    \end{align}
\end{enumerate}
We are given the following specifications:
\begin{enumerate}
    \item Passband edge (which is equal to 
    cutoff frequency), $f_p = f_c$
    \item Stopband edge, $f_s$
    \item Attenuation at stopband edge, $A_s$
    \item Peak-to-peak ripple $\delta$ in the passband.
    It is given in dB and is related to $\epsilon$ as
    \begin{align}
        \delta = 10\log_{10}\brak{1 + \epsilon^2}
        \label{eq:delta-eps}
    \end{align}
\end{enumerate}
and we must find a suitable $n$ and $\epsilon$. From
\eqref{eq:delta-eps},
\begin{align}
    \epsilon = \sqrt{10^{\frac{\delta}{10}} - 1}
    \label{eq:epsilon-del}
\end{align}
At $f_s > f_p = f_c$, using \eqref{eq:chebypol}, $A_s$ is given by
\begin{align}
    A_s = -10\log_{10}\sbrak{1 + \epsilon^2c_n^2\brak{\frac{f_s}{f_p}}} \\
    \implies c_n\brak{\frac{f_s}{f_p}} = \frac{\sqrt{10^{-\frac{A_s}{10}} - 1}}{\epsilon} \\
    \implies n = \frac{\cosh^{-1}\brak{\frac{\sqrt{10^{-\frac{A_s}{10}} - 1}}{\epsilon}}}
    {\cosh^{-1}\brak{\frac{f_s}{f_p}}}
\end{align}
We consider the following specifications:
\begin{enumerate}
    \item Passband edge/cutoff frequency, $f_p = f_c =  {60}{\text{Hz}}$.
    \item Stopband edge, $f_s =  {100}{\text{Hz}}$.
    \item Passband ripple, $\delta =  {0.5}{  \text{dB}}$
    \item Stopband attenuation, $A_s =  {-20}{  \text{dB}}$
\end{enumerate}
$\epsilon = 0.35$ and $n = 3.68$. Hence, we take $n = 4$
as the order of the Chebyshev filter.
\item Design a circuit for your Butterworth filter.
\solution Looking at the table of normalized element values
$L_k$, $C_k$, of the Butterworth filter for order 5, and noting
that de-normalized values $L_k'$ and $C_k'$ are given by
\begin{align}
    C_k' = \frac{C_k}{\omega_c} \qquad L_k' = \frac{L_k}{\omega_c}
\end{align}
De-normalizing these values, taking $f_c = 60$ Hz,
\begin{align}
    C_1' = C_5' =   {1.64}{\text{mF}} \\
    L_2' = L_4' =   {4.29}{\text{mH}} \\
    C_3' =   {5.31}{\text{mF}} \\
\end{align}
The L-C network is shown in Fig. \ref{fig:butter-filter}.
\begin{figure}[!ht]
    \centering
    \begin{circuitikz} 
        \draw (0,0) to[short, o-o] (7,0);
        \draw (0,2) to [short, o-] (1,2) to [L, l=4.29 mH] (3.5,2) to [L, l=4.29 mH] (6,2) to[short, -o] (7,2);
        \draw (1,0) to[C, l=1.64 mF] (1,2);
        \draw (3.5,0) to[C, l=5.31 mF] (3.5,2);
        \draw (6,0) to[C, l=1.64 mF] (6,2);
    \end{circuitikz}
    \caption{L-C Butterworth Filter}
    \label{fig:butter-filter}
\end{figure}
This circuit is simulated in the ngspice code \texttt{codes/5\_3.cir}.
The Python code \texttt{codes/5\_3.py} compares the amplitude response
of the simulated circuit with the theoretical expression.
\item Design a circuit for your Chebyshev filter.
\begin{figure}
    \includegraphics[width=\columnwidth]{figs/5.3.png}
    \caption{Simulation of Butterworth filter.}
    \label{fig:sim-butter}
\end{figure}
\solution Looking at the table of normalized element values
of the Chebyshev filter for order 3 and 0.5 dB ripple,
and de-nomrmalizing those values, taking $f_c =  {50}{  \text{Hz}}$,
\begin{align}
    C_1' =   {4.43}{\text{mF}} \\
    L_2' =   {3.16}{\text{mH}} \\
    C_3' =   {6.28}{\text{mF}} \\
    L_4' =   {2.23}{\text{mH}}
\end{align}
The L-C network is shown in Fig. \ref{fig:cheby-filter}.
\begin{figure}[!ht]
    \centering
    \begin{circuitikz} 
        \draw (0,0) to[short, o-o] (7,0); 
        \draw (1,0) to[C, l=4.43 mF] (1,2);
        \draw (3.5,0) to[C, l=6.28 mF] (3.5,2);
        \draw (0,2) to [short, o-] (1,2) to [L, l=3.16 mH] (3.5,2) to[L, l=2.23 mH] (6,2) to[short, -o] (7,2);
    \end{circuitikz}
    \caption{L-C Chebyshev Filter}
    \label{fig:cheby-filter}
\end{figure}
This circuit is simulated in the ngspice code \texttt{codes/5\_4.cir}.
The Python code \texttt{codes/5\_4.py} compares the amplitude response
of the simulated circuit with the theoretical expression.
\begin{figure}
    \includegraphics[width=\columnwidth]{figs/5.4.png}
    \caption{Simulation of Chebyshev filter.}
    \label{fig:sim-cheby}
\end{figure}
\item Design a low pass digital Butterworth filter for your $H(f)$.
\solution 
\item Design a low pass digital Chebyshev filter for your $H(f)$.
\solution
\end{enumerate}

\end{document}
